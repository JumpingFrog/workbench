\documentclass[a4paper]{article}

\title{MAIG: Exercise Sheet 1}
\author{David Wotherspoon}
\date{}
\begin{document}
\maketitle
\begin{enumerate}
	\item
		\begin{description}
			\item[Tic-Tac-Toe:] Perfect information :- both players have the same view of the board. Deterministic :- no random element.
			\item[Hearts:] Imperfect information :- each player can only see their own hand. Stochastic :- Random element in the shuffling of the cards.
			\item[Halo:] Imperfect information :- player does not know full status of the enemy e.g. health and ammo. Deterministic :- no random element on player's pay-off.
			\item[Age of Empires:] Imperfect information :- "fog of war" means that players may not have the same information. Deterministic :- all actions have one outcome with no random elements.
		\end{description}
	\item
		\begin{enumerate}
			\item black = 5, blue = 4, red = 3, green = 2, silver = 1.
			\item \( 0.3\cdot4 + 0.2\cdot3 + 0.15\cdot2 + 0.2\cdot1 = 2.3\)
			\item p1 = 0.2, p2 = 0.8, expected = 1.6 \newline
			q1 = 0.3, q2 = 0.7, expected = 1.6
			\item \(p1 = \frac{1}{3}, p2 = \frac{2}{3}\)
		\end{enumerate}
	\item
		\begin{enumerate}
			\item Optimal strategy (A2, B1) by elimination of dominated strategies.
			\item No dominant strategies. Nash Equilibrium at (A1, B1) and (A2, B2). Pareto Optimum (A1, B1).
			\item No dominant strategies. Nash Equilibrium (A3, B2). Pareto Optimum (A3, B2)
		\end{enumerate}
	\item
		\begin{enumerate}
			\item
			\begin{description}
				\item[P] \( = \lbrace 1, 2\rbrace \)
				\item[A] \( = \lbrace S, W\rbrace \)
				\item[U] \( = \lbrace U_1, U_2 \rbrace \) \newline
				\( U_1(S, S) = U_2(S, S) = 0 \) \newline
				\( U_1(S, W) = U_2(W, S) = 20 \) \newline
				\( U_1(W, S) = U_2(S, W) = -10 \) \newline
				\( U_1(W, W) = U_2(W, W) = 10 \) \newline
			\end{description}
			\item Optimal strategy is to shirk as it gives a chance of the maximum payoff (20) but also makes the worst case payoff (-10) impossible.
			\item This depends on whether the utility is a true reflection of the student's utility. For example the student may also want to avoid a 0 mark situation.
		\end{enumerate}
\end{enumerate}

\end{document}